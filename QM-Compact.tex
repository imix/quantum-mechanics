\documentclass[10pt,a4paper]{book}
\usepackage[utf8]{inputenc}
\usepackage{amsmath}
\usepackage{amsfonts}
\usepackage{amssymb}
\usepackage{braket}

\title{Quantum Mechanics}
\author{Stefan Aeschbacher}
\date{\today}

\newtheorem{post}{Postulate}

\DeclareMathOperator {\opH} {\hat{H}}
\DeclareMathOperator {\opA} {\hat{A}}
\DeclareMathOperator {\opB} {\hat{B}}
\DeclareMathOperator {\opC} {\hat{C}}

\begin{document}

    \maketitle


    \chapter{Postulates}
    \section{Postulate 1: state}
    
    \begin{post}
    The state of a physical system is described by a state vector that belongs to a complex vector space V, called the state space of the system.
    \end{post}
    
    \chapter{State space}
    \begin{align}
    \ket{ \psi_1} + \ket{ \psi_2} &= \ket{ \psi_3} \\
    \ket{ \psi_1} + \ket{ \psi_2} &= \ket{ \psi_2} + \ket{ \psi_1}
    \end{align}
    \chapter{Operators}
    An operator acting on a ket creates a new ket:
    \begin{align}
    \opA \ket{\psi} = \ket{\psi'} \\
    \ket{\psi}, \ket{\psi'} \in V
    \end{align}
    Operators are linear
    \begin{align}
    \opA (a_1\ket{\psi_1} + a_2\ket{\psi_2}) = (a_1\opA\ket{\psi_1} + a_2\opA\ket{\psi_2}) \\
    \ket{\psi}, \ket{\psi'} \in V; a_1, a_2 \in \mathbb{C}
    \end{align}
    Operators are associative and commutative under addition
    \begin{align}
    \opA + (\opB + \opC) = (\opA + \opB) + \opC \\
    \opA + \opB = \opB + \opA 
    \end{align}
    Multiplying operators is interpreted as applying them to kets. It is associative but NOT (in general) commutative.
    \begin{align}
    \opA\opB\ket{\psi} = \opA(\opB\ket{\psi}) = \opA\ket{\psi'} \\
    \opA(\opB\opC) = (\opA\opB)\opC \\
    \opA\opB \ne \opB\opA
    \end{align}
    The lack of commutativeness makes the "commutator" useful
    \begin{align}
    [\opA, \opB] = \opA\opB - \opB\opA \\
    \end{align}    
    \section{Projection Operators}
    A projection operator is defined
    \begin{align}
    [\opA, \opB] = \opA\opB - \opB\opA \\
    \end{align}  
    \chapter{Eigenvectors and Eigenvalues}
    \section{Degenerate Eigenvectors}
    If two eigenvectors have the same eigenvalue:
    \begin{align}
    \opH\ket{\lambda_1} = \lambda \ket{\lambda_1} \\
    \opH\ket{\lambda_2} = \lambda \ket{\lambda_2} 
    \end{align}
    their linear combination is an eigenvector as well:
    \begin{align}
    \alpha \opH\ket{\lambda_1} = \lambda \alpha \ket{\lambda_1} \\
    \beta \opH\ket{\lambda_1} = \lambda \beta \ket{\lambda_1} \\
    \opH[\alpha \ket{\lambda_1} + \beta \ket{\lambda_2}] = \lambda [\alpha \ket{\lambda_1} + \beta \ket{\lambda_2}]    
    \end{align}
    Therefore it is possible to create two orthogonal eigenvectors for this eigenvalue.
    
    The probability in the degenerate case is the sum of the probabilities for each eigenvector $| \bra{\opH} \ket{\lambda_1}|^2 + | \bra{\opH} \ket{\lambda_2}|^2$
    
\end{document}